%%%%%%%%%%%%%%%%%%%%%%%%%%%%%%%%%%%%%%%%%%%%%%%%%%%%%%%%%%%%%%%%%%%%%%%%%%
%Author:																 %
%-------																 %
%Yannis Baehni at University of Zurich									 %
%baehni.yannis@uzh.ch													 %
%																		 %
%Version log:															 %
%------------															 %
%06/02/16 . Basic structure												 %
%04/08/16 . Layout changes including section, contents, abstract.		 %
%05/11/16 . Simon, name changes%
% more simon changes 5.4.17 %
%%%%%%%%%%%%%%%%%%%%%%%%%%%%%%%%%%%%%%%%%%%%%%%%%%%%%%%%%%%%%%%%%%%%%%%%%%

%Page Setup
\documentclass[
	11pt, 
	oneside, 
	a4paper,
	reqno,
	final
]{amsart}

\usepackage{commath}

\usepackage{fouridx}

\usepackage{tikz-cd}

\usepackage[
	left = 3cm, 
	right = 3cm, 
	top = 3cm, 
	bottom = 3cm
]{geometry}

%Headers and footers
\usepackage{fancyhdr}
	\pagestyle{fancy}
	%Clear fields
	\fancyhf{}
	%Header right
	\fancyhead[R]{
		\footnotesize
		Simon Gr\"uning\\
		\href{mailto:simon.gruening@uzh.ch}{simon.gruening@uzh.ch}
	}
	%Header left
	\fancyhead[L]{
		\footnotesize
		ETCS\\
		HS17
	}
	%Page numbering in footer
	\fancyfoot[C]{\thepage}
	%Separation line header and footer
	\renewcommand{\headrulewidth}{0.4pt}
	%\renewcommand{\footrulewidth}{0.4pt}
	
	\setlength{\headheight}{19pt} 

%Title
\usepackage[foot]{amsaddr}
\usepackage{xspace}
\makeatletter
\def\@textbottom{\vskip \z@ \@plus 1pt}
\let\@texttop\relax
\usepackage{etoolbox}
\patchcmd{\abstract}{\scshape\abstractname}{\textbf{\abstractname}}{}{}

%Switching commands for different section formats
%Mainsectionsytle
\newcommand{\mainsectionstyle}{%
  	\renewcommand{\@secnumfont}{\bfseries}
  	\renewcommand\section{\@startsection{section}{1}%
    	\z@{.5\linespacing\@plus.7\linespacing}{-.5em}%
    	{\normalfont\bfseries}}%
	\renewcommand\subsection{\@startsection{subsection}{2}%
    	\z@{.5\linespacing\@plus.7\linespacing}{-.5em}%
    	{\normalfont\bfseries}}%
	\renewcommand\subsubsection{\@startsection{subsubsection}{3}%
    	\z@{.5\linespacing\@plus.7\linespacing}{-.5em}%
    	{\normalfont\bfseries}}%
}
\newcommand{\originalsectionstyle}{%
\def\@secnumfont{\bfseries}%\mdseries
\def\section{\@startsection{section}{1}%
  \z@{.7\linespacing\@plus\linespacing}{.5\linespacing}%
  {\normalfont\bfseries\centering}}
}
%Formatting title of TOC
\renewcommand{\contentsnamefont}{\bfseries}
%Table of Contents
\setcounter{tocdepth}{3}

% Add bold to \section titles in ToC and remove . after numbers
\renewcommand{\tocsection}[3]{%
  \indentlabel{\@ifnotempty{#2}{\bfseries\ignorespaces#1 #2\quad}}\bfseries#3}
% Remove . after numbers in \subsection
\renewcommand{\tocsubsection}[3]{%
  \indentlabel{\@ifnotempty{#2}{\ignorespaces#1 #2\quad}}#3}
\let\tocsubsubsection\tocsubsection% Update for \subsubsection
%...

\newcommand\@dotsep{4.5}
\def\@tocline#1#2#3#4#5#6#7{\relax
  \ifnum #1>\c@tocdepth % then omit
  \else
    \par \addpenalty\@secpenalty\addvspace{#2}%
    \begingroup \hyphenpenalty\@M
    \@ifempty{#4}{%
      \@tempdima\csname r@tocindent\number#1\endcsname\relax
    }{%
      \@tempdima#4\relax
    }%
    \parindent\z@ \leftskip#3\relax \advance\leftskip\@tempdima\relax
    \rightskip\@pnumwidth plus1em \parfillskip-\@pnumwidth
    #5\leavevmode\hskip-\@tempdima{#6}\nobreak
    \leaders\hbox{$\m@th\mkern \@dotsep mu\hbox{.}\mkern \@dotsep mu$}\hfill
    \nobreak
    \hbox to\@pnumwidth{\@tocpagenum{\ifnum#1=1\bfseries\fi#7}}\par% <-- \bfseries for \section page
    \nobreak
    \endgroup
  \fi}
\AtBeginDocument{%
\expandafter\renewcommand\csname r@tocindent0\endcsname{0pt}
}
\def\l@subsection{\@tocline{2}{0pt}{2.5pc}{5pc}{}}
\def\l@subsubsection{\@tocline{2}{0pt}{4.5pc}{5pc}{}}
\makeatother

\advance\footskip0.4cm
\textheight=54pc    %a4paper
\textheight=50.5pc %letterpaper
\advance\textheight-0.4cm
\calclayout

%Font settings
%\usepackage{anyfontsize}
%Footnote settings
%\usepackage{mathptmx}
\usepackage{footmisc}
%	\renewcommand*{\thefootnote}{\fnsymbol{footnote}}

%Further math environments
%Further math fonts (loads amsfonts implicitely)
\usepackage{amssymb}
%Redefinition of \text
%\usepackage{amstext}
\usepackage{upref}
%Graphics
%\usepackage{graphicx}
%\usepackage{caption}
%\usepackage{subcaption}
%Frames
\usepackage{mdframed}
\allowdisplaybreaks
%\usepackage{interval}
\newcommand{\toup}{%
  \mathrel{\nonscript\mkern-1.2mu\mkern1.2mu{\uparrow}}%
}
\newcommand{\todown}{%
  \mathrel{\nonscript\mkern-1.2mu\mkern1.2mu{\downarrow}}%
}
\AtBeginDocument{\renewcommand*\d{\mathop{}\!\mathrm{d}}}
\renewcommand{\Re}{\operatorname{Re}}
\renewcommand{\Im}{\operatorname{Im}}
\DeclareMathOperator\Log{Log}
\DeclareMathOperator\Arg{Arg}
\DeclareMathOperator\sech{sech}

\DeclareMathOperator*\esssup{ess.sup}

%\usepackage{hhline}
%\usepackage{booktabs} 
%\usepackage{array}
%\usepackage{xfrac} 
%\everymath{\displaystyle}
%Enumerate
\usepackage{tikz}
\usetikzlibrary{patterns}
\pgfdeclarepatternformonly{adjusted lines}{\pgfqpoint{-1pt}{-1pt}}{\pgfqpoint{40pt}{40pt}}{\pgfqpoint{39pt}{39pt}}%
{
  \pgfsetlinewidth{.8pt}
  \pgfpathmoveto{\pgfqpoint{0pt}{0pt}}
  \pgfpathlineto{\pgfqpoint{39.1pt}{39.1pt}}
  \pgfusepath{stroke}
}
\usepackage{enumitem} 
%\renewcommand{\labelitemi}{$\bullet$}
%\renewcommand{\labelitemii}{$\ast$}
%\renewcommand{\labelitemiii}{$\cdot$}
%\renewcommand{\labelitemiv}{$\circ$}
%Colors
%\usepackage{color}
%\usepackage[cmtip, all]{xy}
%Theorems
\newtheoremstyle{bold}              	 %Name
  {}                                     %Space above
  {}                                     %Space below
  {\itshape}		                     %Body font
  {}                                     %Indent amount
  {\scshape}                             %Theorem head font
  {.}                                    %Punctuation after theorem head
  { }                                    %Space after theorem head, ' ', 
  										 %	or \newline
  {} 
\theoremstyle{bold}
\newtheorem*{definition*}{Definition}
\newtheorem{definition}{Definition}[section]
\newtheorem*{lemma*}{Lemma}
\newtheorem{lemma}{Lemma}[section]
\newtheorem{Proof}{Proof}[section]
\newtheorem{proposition}{Proposition}[section]
\newtheorem{properties}{Properties}[section]
\newtheorem{corollary}{Corollary}[section]
\newtheorem*{theorem*}{Theorem}
\newtheorem{theorem}{Theorem}[section]
\newtheorem{example}{Example}[section]
\newtheorem*{remark*}{Remark}
\newtheorem{remark}{Remark}[section]
%German non-ASCII-Characters
%Graphics-Tool
%\usepackage{tikz}
%\usepackage{tikzscale}
%\usepackage{bbm}
%\usepackage{bera}
%Listing-Setup
%Bibliographie
\usepackage[backend=bibtex, style=alphabetic]{biblatex}
%\usepackage[babel, german = swiss]{csquotes}
\bibliography{Bibliography}
%PDF-Linking
%\usepackage[hyphens]{url}
\usepackage[bookmarksopen=true,bookmarksnumbered=true]{hyperref}
%\PassOptionsToPackage{hyphens}{url}\usepackage{hyperref}
\hypersetup{
  colorlinks   = true, %Colours links instead of ugly boxes
  urlcolor     = blue, %Colour for external hyperlinks
  linkcolor    = blue, %Colour of internal links
  citecolor    = blue %Colour of citations
}
%Weierstrass-P symbol for power set
\newcommand{\powerset}{\raisebox{.15\baselineskip}{\Large\ensuremath{\wp}}}

\usepackage[utf8]{inputenc}
\usepackage[english]{babel}
\usepackage{minted}
\usemintedstyle{pastie}

% Xy-pic for graphs woo
\usepackage{xy}
%\usepackage{xypic}
\input xy
\xyoption{all}

%for lightning weee
\usepackage{ stmaryrd }


\begin{document}

\title{Elementary Theory of the Category of Sets}
\author{Simon Gr\"uning}
\address[Simon Gr\"uning]{University of Zurich, R\"{a}mistrasse 71, 8006 Zurich}
\email[Simon Gr\"uning]{\href{mailto:simon.gruening@uzh.ch}{simon.gruening@uzh.ch}}

\newtheorem{axiom}{Axiom}
\setcounter{axiom}{-1}

\maketitle

\section*{(Category Theory Exam)}


\clearpage




\section{The Axioms}

\begin{axiom}[Category Theory]
All Axioms of Category Theory hold.
\end{axiom}

\begin{remark}
In the following we will be working in a category for which all the axioms stated up to this point hold. We use the terms object/set and function/morphism interchangeably.

We use the notation for composition of morphisms: $a \circ b = ba$.
\end{remark}

\begin{axiom}[Completeness]
The category is complete and cocomplete.
\end{axiom}

\begin{remark}
The existence of all finite limits guarantees the existence of a terminal object $\mathbf{1}$, a product $\times$, and the equalizer. Dually, we have the existence of the initial object $\mathbf{0}$, the coproduct $+$, and the coequalizer. We also have the existence of the inverse image.
\end{remark}

\begin{remark}
$\mathbf{1}$ plays the role of the one-element set, as for any set $X$ there is precisely one map $t: X \longrightarrow \mathbf{1}$. Since terminal sets are unique up to isomorphism, we may fix one and speak of "the" terminal object. We do this for all other similar cases, overloading the meaning of "the".
\end{remark}

\begin{definition}
Let X be a set, $x: \mathbf{1} \longrightarrow X$ a function. We call $x$ an \textbf{element} of $X$ and write $x \in X$. For a function $f:X \longrightarrow Y$ we define the \textbf{evaluation} as a special case of composition 
\begin{equation*}
f(x) := f \circ x: \mathbf{1} \longrightarrow Y.
\end{equation*}
Notice that $f(x) \in Y$.
\end{definition}

\begin{definition}[$\dagger$]
Let $f: X \longrightarrow Y$ and $y \in Y$. In our specific case, we define the \textbf{Inverse Image} of $y$ under $f$ to be an object $A$ and a function $j:A \longrightarrow X$ such that:
\begin{enumerate}
\item $\forall a \in A: f(j(a)) = y$. Thus the following diagram must commute:

\begin{equation*}
\begin{tikzcd}
A \arrow[d, "j"]
& \mathbf{1} \arrow[d, "y"] \arrow[l, "a", swap] \\
 X \arrow[r, "f"]
& Y
\end{tikzcd}
\end{equation*}

\item For all objects $I$ and functions $q: I \longrightarrow X$ such that $\forall t \in I: f(q(t)) = y$ there exists a unique function $\bar{q}: I \longrightarrow A$ such that $q = j \circ \bar{q}$. (?!ToDo!?): 

\begin{equation*}
\begin{tikzcd}
I
\arrow[drr, bend left, ""]
\arrow[ddr, bend right, "q"]
\arrow[dr, dotted, "{\bar{q}}" description] & & \\
& A \arrow[r, ""] \arrow[d, "j"]
& \mathbf{1} \arrow[d, "y"] \\
& X \arrow[r, "f"]
& Y
\end{tikzcd}
\end{equation*}


\end{enumerate}
\end{definition}

\begin{axiom}[Functions as Set]
For every pair of sets $X, Y$, the exponential $Y^X$ exists.
\end{axiom}

\begin{remark}[$\dagger$]
The object $Y^X$ plays the role of an internal hom$[X,Y]$, in this case it wants to be the function set. For any fixed set $B$ we have an adjunction $(\-- \times B) \dashv (\--)^B$ and thus a natural bijection for any two other sets $A,C$:
\begin{equation*}
Hom(A \times B, C) \simeq Hom(A, C^B).
\end{equation*}

By Lawvere: For any two objects $A, B$ there exists an object $B^A$ and a mapping $A \times B^A \xrightarrow{e} B$ with the property that for any object $X$ and any mapping $A \times X\xrightarrow{f} B$ there is a unique mapping $X \xrightarrow{h} B^A$ such that $(\mathbf{1}_A \times h)e = f$. We call $e$ the evaluation map and we have that for $a \in A, f:A\longrightarrow B$, we may evaluate the name $\fourIdx{\ulcorner\mkern-3mu}{}{\urcorner}{}{f} \in B^A$ as $(a,\fourIdx{\ulcorner\mkern-3mu}{}{\urcorner}{}{f})e = af$.

\end{remark}




\begin{definition}
A \textbf{natural number system} is a tuple $(N,0,s)$ with $N$ an object, $0 \in N$, and a $s: N \longrightarrow N$ such that for any object $X$, $a \in X$, and $r: X \longrightarrow X$
there is a unique $x: N \longrightarrow X$ such that the following diagram commutes:

\begin{equation*}
\begin{tikzcd}
\mathbf{1} \arrow[r, "0"] \arrow[d, "id_\mathbf{1}"]
& N \arrow[r, "s"] \arrow[d, "x", dotted]
& N \arrow[d, "x", dotted]   \\
\mathbf{1} \arrow[r, "a"]
& X \arrow[r, "r"]
& X
\end{tikzcd}.
\end{equation*}

\end{definition}

\begin{axiom}[Natural Numbers]
There exists a natural number system (Dedekind-Pierce Object).
\end{axiom}

\begin{definition}
A \textbf{generator} in a category $\mathcal{C}$ is an object $G$ such that for any two morphisms $f,g: X \longrightarrow Y$ in $\mathcal{C}$ if $f \neq g$ there exists a morphism $h: G \longrightarrow X$ such that $f \circ h \neq g \circ h$
\end{definition}

\begin{axiom}[Equality of Functions]
$\mathbf{1}$ is a generator.
\end{axiom}

\begin{remark}
Since $\mathbf{1}$ is used to represent our elements, it follows that two functions are equal precisely when they have the same domain and codomain, and they are the same on every element. To see this let the morphism $h$ be denoted as element $x \in X$. Then
\begin{equation*}
\forall f,g: X \longrightarrow Y: f \neq g \implies \exists x \in X : f(x) = f \circ x \neq g \circ x = g(x).
\end{equation*}

Further it follows that if an object $A$ has precisely one element then $A = \mathbf{1}$. (ToDo)
\end{remark}

\begin{axiom}[AC]
For any morphism $f$, if there exists an $x \in dom(f)$ then there exists a quasi-inverse $g$ such that $fgf = f$.
\end{axiom}

\begin{remark}
All other Axioms (even the remaining three) hold in the category $\mathcal{P}os$ of partially ordered sets and order-preserving maps, however the Axiom of Choice does not. It is thus independent of the other Axioms by model construction. 
\end{remark}

\begin{definition}
$a:X \longrightarrow A$ is a \textbf{subset} of $A$ if $a$ is a monomorphism.

$x$ is a \textbf{member} of $a$ if for some $A$, $x \in A$, $a$ is a subset of $A$, and there exists $\bar{x}$ such that $\bar{x}a = x$. In this case we also write $x \in a$.

We write $a \subseteq b$ if for some $A$, $a$ and $b$ are both subsets of $A$ and there exists $h$ such that $a = hb$ ie. $a$ factors over $b$.
\end{definition}

\begin{theorem}
Let $a, b$ subsets of $A$. Then
\begin{equation}
a \subseteq b \iff \forall x \in A: x \in a \implies x \in b
\end{equation}
\end{theorem}

\begin{proof}
\begin{enumerate} 
\item $\Rightarrow :$ Let $a \subseteq b$ and $x \in A$ with $x \in a$. Then by definition we find a morphism $h$ and $\bar{x}$ such that their respective triangles commute:

\begin{equation}
\begin{tikzcd}[column sep=small]
& \mathbf{1} \arrow[dl, "\bar{x}"] \arrow[dr, dotted] \arrow[dd, "x", bend right=90,looseness=2,swap]& \\
\Box \arrow[rr, "h"] \arrow[dr, "a", hook] &     & \Box \arrow[dl, "b", hook] \\
& A
\end{tikzcd}
\end{equation} \newline

It follows that $x = \bar{x}a = \bar{x}(hb) = (\bar{x}h)b$ is our sought after factoring for $x \in b$.

\item $\Leftarrow :$ Let $a \in A$ and $a,b: \Box \longrightarrow A$ two monomorphisms. Then the Axiom of Choice implies that $\exists g: A \longrightarrow \Box : bgb = b$. By the left cancelative property of our monomorphism $b$, we retrieve $bg = id_\Box$. Define $h := ag$. We want to show that $a = hb = agb$. By Axiom 4 we may do so by proving that $\forall \bar{x} \in \Box: \bar{x}a = \bar{x}agb$. Fixing an arbitrary $\bar{x}$, we find that $x := \bar{x}a$ satisfies $x \in A$ with $x \in a$, thus $x \in b$ and it follows that $\exists y : x = yb$. Then
\begin{equation*}
\bar{x}hb = \bar{x}agb = xgb = (yb)gb = yb = x = \bar{x}a.
\end{equation*}
Thus $hb = a$ is our factoring for $a \subseteq b$. 
\end{enumerate}
\end{proof}

%%%

\begin{axiom}[Empty Set]
Each object other than $\mathbf{0}$ has elements.
\end{axiom}

\begin{axiom}[Disjoint Union]
Each $x \in A + B$ is a member of one of the injections, ie. $x$ factors over one of the two coproduct inclusions
\end{axiom}

\begin{axiom}[Larger Sets]
There is an object with more than one element.
\end{axiom}

\begin{lemma}
$\mathbf{0}$ has no elements.
\end{lemma}

\begin{proof}
If $\mathbf{0}$ had an element, then we have a morphism $\mathbf{1} \longrightarrow \mathbf{0}$ which when precomposed with the unique map $\mathbf{0} \longrightarrow \mathbf{1}$  gives us by uniqueness the map $id_\mathbf{0}:\mathbf{0} \longrightarrow \mathbf{0}$. It follows that $\mathbf{0} = \mathbf{1}$ which contradicts Axiom 8 since then every object would have precisely one element.
\end{proof}

\begin{remark}
This group of axioms also helps us create a well-defined $\mathbf{2} := \mathbf{1}+\mathbf{1}$ set.
\end{remark}

%%%

\section{Peano Axioms Hold}

\begin{theorem}[Peano's 7th postulate]
The successor function $s$ is injective.
\end{theorem}

\begin{proof}
needs: predecessor, injective, primitive recursion, 2, 
\end{proof}

\begin{theorem}[Peano's 9th postulate]
Peano's Axiom of Induction holds for N.
\end{theorem}
needs: subset, inverse image, simple recursion
\begin{proof}

\end{proof}

\begin{remark}
Peano's other Postulates hold implicitly (ToDo).
\end{remark}

\section{Meta}

\begin{theorem}
If $\mathcal{C}$ is a locally small, complete model of ETCS, then $\mathcal{C}$ is equivalent to $Set$.
\end{theorem}


%%%%%%%%%%%%%%%%%%%%%%%%%


\clearpage

\begin{appendix}

\section{Small Things}

\begin{remark}
We say a morphism f factors over g if if there exists h such that $h \circ g = f$.
\end{remark}


\section{Alternate Axiom Phrasings}

\setcounter{axiom}{0}


\begin{axiom}[Associativity and Identity, Leinster]
For all sets W, X, Y, Z and functions
\begin{equation*}
\begin{tikzcd}
W \arrow[r,"f"] & X \arrow[r,"g"] & Y \arrow[r,"h"] & Z 
\end{tikzcd}
\end{equation*}
we have $h \circ (g \circ f) = (h \circ g) \circ f$. 

Furthermore, for all sets $X$, there exists a function $1_{X}: X \longrightarrow X$ such that for all functions $g: X \longrightarrow Y$, $f: W \longrightarrow X$, we have that $g \circ 1_{X} = g$ and $1_{X} \circ f = f$.
\end{axiom}

\begin{axiom}[Elements, Leinster]
There exists a terminal set $T$.
\end{axiom}

\begin{axiom}[Empty Set, Leinster]
There exists a set with no elements.
\end{axiom}

\begin{axiom}[Cartesian Product, Leinster]
Every pair of sets $X, Y$ has a product $(X \times Y, p_1^{X,Y}, p_2^{X,Y})$.
\end{axiom}

\begin{axiom}[Equality of Functions, Leinster]
Let $f,g: X \longrightarrow Y$ two functions between sets. If $\forall x \in X: f(x) = g(x)$ then $f = g$.
\end{axiom}

\begin{axiom}[Inverse Image, Leinster]
For every function $f: X \longrightarrow Y$, $y \in Y$, there exists an inverse image $f^{-1}(y)$ of $y$ under $f$.
\end{axiom}

\begin{axiom}[Subsets, Leinster]
There exists a subset classifier.
\end{axiom}

\begin{axiom}[Choice, Leinster]
Every surjection has a right inverse.
\end{axiom}

\begin{axiom}[Replacement, Leinster]
TODO equality to ZFC
\end{axiom}

\section{Sources}

\begin{enumerate}
\item AN ELEMENTARY THEORY
OF THE CATEGORY OF SETS (LONG VERSION)
WITH COMMENTARY - F. WILLIAM LAWVERE
\item Rethinking set theory - Tome Leinster
\item \href{https://golem.ph.utexas.edu/category/2014/01/an_elementary_theory_of_the_ca.html}{n-Category Cafe} - An Elementary Theory of the Category of Sets by Clive Newstead
\item ncatlab.org
\end{enumerate}

\end{appendix}

\end{document}
