%%%%%%%%%%%%%%%%%%%%%%%%%%%%%%%%%%%%%%%%%%%%%%%%%%%%%%%%%%%%%%%%%%%%%%%%%%
%Author:																 %
%-------																 %
%Yannis Baehni at University of Zurich									 %
%baehni.yannis@uzh.ch													 %
%																		 %
%Version log:															 %
%------------															 %
%06/02/16 . Basic structure												 %
%04/08/16 . Layout changes including section, contents, abstract.		 %
%05/11/16 . Simon, name changes%
% more simon changes 5.4.17 %
%%%%%%%%%%%%%%%%%%%%%%%%%%%%%%%%%%%%%%%%%%%%%%%%%%%%%%%%%%%%%%%%%%%%%%%%%%

%Page Setup
\documentclass[
	11pt, 
	oneside, 
	a4paper,
	reqno,
	final
]{amsart}

\usepackage{commath}

\usepackage{fouridx}

\usepackage{tikz-cd}

\usepackage[
	left = 3cm, 
	right = 3cm, 
	top = 3cm, 
	bottom = 3cm
]{geometry}

%Headers and footers
\usepackage{fancyhdr}
	\pagestyle{fancy}
	%Clear fields
	\fancyhf{}
	%Header right
	\fancyhead[R]{
		\footnotesize
		Simon Gr\"uning\\
		\href{mailto:simon.gruening@uzh.ch}{simon.gruening@uzh.ch}
	}
	%Header left
	\fancyhead[L]{
		\footnotesize
		ETCS\\
		HS17
	}
	%Page numbering in footer
	\fancyfoot[C]{\thepage}
	%Separation line header and footer
	\renewcommand{\headrulewidth}{0.4pt}
	%\renewcommand{\footrulewidth}{0.4pt}
	
	\setlength{\headheight}{19pt} 

%Title
\usepackage[foot]{amsaddr}
\usepackage{xspace}
\makeatletter
\def\@textbottom{\vskip \z@ \@plus 1pt}
\let\@texttop\relax
\usepackage{etoolbox}
\patchcmd{\abstract}{\scshape\abstractname}{\textbf{\abstractname}}{}{}

%Switching commands for different section formats
%Mainsectionsytle
\newcommand{\mainsectionstyle}{%
  	\renewcommand{\@secnumfont}{\bfseries}
  	\renewcommand\section{\@startsection{section}{1}%
    	\z@{.5\linespacing\@plus.7\linespacing}{-.5em}%
    	{\normalfont\bfseries}}%
	\renewcommand\subsection{\@startsection{subsection}{2}%
    	\z@{.5\linespacing\@plus.7\linespacing}{-.5em}%
    	{\normalfont\bfseries}}%
	\renewcommand\subsubsection{\@startsection{subsubsection}{3}%
    	\z@{.5\linespacing\@plus.7\linespacing}{-.5em}%
    	{\normalfont\bfseries}}%
}
\newcommand{\originalsectionstyle}{%
\def\@secnumfont{\bfseries}%\mdseries
\def\section{\@startsection{section}{1}%
  \z@{.7\linespacing\@plus\linespacing}{.5\linespacing}%
  {\normalfont\bfseries\centering}}
}
%Formatting title of TOC
\renewcommand{\contentsnamefont}{\bfseries}
%Table of Contents
\setcounter{tocdepth}{3}

% Add bold to \section titles in ToC and remove . after numbers
\renewcommand{\tocsection}[3]{%
  \indentlabel{\@ifnotempty{#2}{\bfseries\ignorespaces#1 #2\quad}}\bfseries#3}
% Remove . after numbers in \subsection
\renewcommand{\tocsubsection}[3]{%
  \indentlabel{\@ifnotempty{#2}{\ignorespaces#1 #2\quad}}#3}
\let\tocsubsubsection\tocsubsection% Update for \subsubsection
%...

\newcommand\@dotsep{4.5}
\def\@tocline#1#2#3#4#5#6#7{\relax
  \ifnum #1>\c@tocdepth % then omit
  \else
    \par \addpenalty\@secpenalty\addvspace{#2}%
    \begingroup \hyphenpenalty\@M
    \@ifempty{#4}{%
      \@tempdima\csname r@tocindent\number#1\endcsname\relax
    }{%
      \@tempdima#4\relax
    }%
    \parindent\z@ \leftskip#3\relax \advance\leftskip\@tempdima\relax
    \rightskip\@pnumwidth plus1em \parfillskip-\@pnumwidth
    #5\leavevmode\hskip-\@tempdima{#6}\nobreak
    \leaders\hbox{$\m@th\mkern \@dotsep mu\hbox{.}\mkern \@dotsep mu$}\hfill
    \nobreak
    \hbox to\@pnumwidth{\@tocpagenum{\ifnum#1=1\bfseries\fi#7}}\par% <-- \bfseries for \section page
    \nobreak
    \endgroup
  \fi}
\AtBeginDocument{%
\expandafter\renewcommand\csname r@tocindent0\endcsname{0pt}
}
\def\l@subsection{\@tocline{2}{0pt}{2.5pc}{5pc}{}}
\def\l@subsubsection{\@tocline{2}{0pt}{4.5pc}{5pc}{}}
\makeatother

\advance\footskip0.4cm
\textheight=54pc    %a4paper
\textheight=50.5pc %letterpaper
\advance\textheight-0.4cm
\calclayout

%Font settings
%\usepackage{anyfontsize}
%Footnote settings
%\usepackage{mathptmx}
\usepackage{footmisc}
%	\renewcommand*{\thefootnote}{\fnsymbol{footnote}}

%Further math environments
%Further math fonts (loads amsfonts implicitely)
\usepackage{amssymb}
%Redefinition of \text
%\usepackage{amstext}
\usepackage{upref}
%Graphics
%\usepackage{graphicx}
%\usepackage{caption}
%\usepackage{subcaption}
%Frames
\usepackage{mdframed}
\allowdisplaybreaks
%\usepackage{interval}
\newcommand{\toup}{%
  \mathrel{\nonscript\mkern-1.2mu\mkern1.2mu{\uparrow}}%
}
\newcommand{\todown}{%
  \mathrel{\nonscript\mkern-1.2mu\mkern1.2mu{\downarrow}}%
}
\AtBeginDocument{\renewcommand*\d{\mathop{}\!\mathrm{d}}}
\renewcommand{\Re}{\operatorname{Re}}
\renewcommand{\Im}{\operatorname{Im}}
\DeclareMathOperator\Log{Log}
\DeclareMathOperator\Arg{Arg}
\DeclareMathOperator\sech{sech}

\DeclareMathOperator*\esssup{ess.sup}

%\usepackage{hhline}
%\usepackage{booktabs} 
%\usepackage{array}
%\usepackage{xfrac} 
%\everymath{\displaystyle}
%Enumerate
\usepackage{tikz}
\usetikzlibrary{patterns}
\pgfdeclarepatternformonly{adjusted lines}{\pgfqpoint{-1pt}{-1pt}}{\pgfqpoint{40pt}{40pt}}{\pgfqpoint{39pt}{39pt}}%
{
  \pgfsetlinewidth{.8pt}
  \pgfpathmoveto{\pgfqpoint{0pt}{0pt}}
  \pgfpathlineto{\pgfqpoint{39.1pt}{39.1pt}}
  \pgfusepath{stroke}
}
\usepackage{enumitem} 
%\renewcommand{\labelitemi}{$\bullet$}
%\renewcommand{\labelitemii}{$\ast$}
%\renewcommand{\labelitemiii}{$\cdot$}
%\renewcommand{\labelitemiv}{$\circ$}
%Colors
%\usepackage{color}
%\usepackage[cmtip, all]{xy}
%Theorems
\newtheoremstyle{bold}              	 %Name
  {}                                     %Space above
  {}                                     %Space below
  {\itshape}		                     %Body font
  {}                                     %Indent amount
  {\scshape}                             %Theorem head font
  {.}                                    %Punctuation after theorem head
  { }                                    %Space after theorem head, ' ', 
  										 %	or \newline
  {} 
\theoremstyle{bold}
\newtheorem*{definition*}{Definition}
\newtheorem{definition}{Definition}[section]
\newtheorem*{lemma*}{Lemma}
\newtheorem{lemma}{Lemma}[section]
\newtheorem{Proof}{Proof}[section]
\newtheorem{proposition}{Proposition}[section]
\newtheorem{properties}{Properties}[section]
\newtheorem{corollary}{Corollary}[section]
\newtheorem*{theorem*}{Theorem}
\newtheorem{theorem}{Theorem}[section]
\newtheorem{example}{Example}[section]
\newtheorem*{remark*}{Remark}
\newtheorem{remark}{Remark}[section]
%German non-ASCII-Characters
%Graphics-Tool
%\usepackage{tikz}
%\usepackage{tikzscale}
%\usepackage{bbm}
%\usepackage{bera}
%Listing-Setup
%Bibliographie
\usepackage[backend=bibtex, style=alphabetic]{biblatex}
%\usepackage[babel, german = swiss]{csquotes}
\bibliography{Bibliography}
%PDF-Linking
%\usepackage[hyphens]{url}
\usepackage[bookmarksopen=true,bookmarksnumbered=true]{hyperref}
%\PassOptionsToPackage{hyphens}{url}\usepackage{hyperref}
\hypersetup{
  colorlinks   = true, %Colours links instead of ugly boxes
  urlcolor     = blue, %Colour for external hyperlinks
  linkcolor    = blue, %Colour of internal links
  citecolor    = blue %Colour of citations
}
%Weierstrass-P symbol for power set
\newcommand{\powerset}{\raisebox{.15\baselineskip}{\Large\ensuremath{\wp}}}

\usepackage[utf8]{inputenc}
\usepackage[english]{babel}
\usepackage{minted}
\usemintedstyle{pastie}

% Xy-pic for graphs woo
\usepackage{xy}
%\usepackage{xypic}
\input xy
\xyoption{all}

%for lightning weee
\usepackage{ stmaryrd }


\begin{document}

\title{Elementary Theory of the Category of Sets}
\author{Simon Gr\"uning}
\address[Simon Gr\"uning]{University of Zurich, R\"{a}mistrasse 71, 8006 Zurich}
\email[Simon Gr\"uning]{\href{mailto:simon.gruening@uzh.ch}{simon.gruening@uzh.ch}}

\newtheorem{axiom}{Axiom}
\setcounter{axiom}{-1}

\maketitle

\section*{(Category Theory Exam)}


\clearpage


\section{Leinster's Axioms}

\begin{remark}
In the following we will be working in a theory for which all the axioms stated up to this point hold. Our axioms will apply to the following data:

\begin{enumerate}
\item \textbf{Things} we call sets
\item \textbf{Processes} between the sets which we call functions and denote $f: X \longrightarrow Y$
\item \textbf{Composition} of two functions $f: X \longrightarrow Y$, $g: Y \longrightarrow Z$, which we denote as $g \circ f: X \longrightarrow Z$.
\end{enumerate}

We use the terms object/set and function/morphism interchangeably. We use the spatially natural notation for composition of morphisms: $a \circ b = ba$.

Rather than defining our theory through an inclusion $\in$, we will try to examine how the sets behave with respect to one another. We probe them for wanted properties and leave the rest to the specific model. Leinster argues that the the axioms of ZFC are not very transparent and natural. When working with sets, we know how they behave and which actions are legal, however we rarely base this on the axioms. This alternate construction may remedy this.
\end{remark}

\begin{axiom}[Associativity and Identity]
For all sets W, X, Y, Z and functions
\begin{equation*}
\begin{tikzcd}
W \arrow[r,"f"] & X \arrow[r,"g"] & Y \arrow[r,"h"] & Z 
\end{tikzcd}
\end{equation*}
we have $h \circ (g \circ f) = (h \circ g) \circ f$. 

Furthermore, for all sets $X$, there exists a function $1_{X}: X \longrightarrow X$ such that for all functions $g: X \longrightarrow Y$, $f: W \longrightarrow X$, we have that $g \circ 1_{X} = g$ and $1_{X} \circ f = f$.
\end{axiom}

\begin{remark}
This Axiom tells us that we can begin with a category, and we shall further restrict the behaviour of this category in the following axioms until we have enough "essence" for a well-defined set theory.
\end{remark}

\begin{axiom}[Elements]
There exists a terminal set $\mathbf{1}$.
\end{axiom}

\begin{remark}
$\mathbf{1}$ plays the role of the one-element set, as for any set $X$ there is precisely one map $t: X \longrightarrow \mathbf{1}$. Since terminal sets are unique up to isomorphism, we may fix one and speak of "the" terminal object. We do this for all other similar cases, overloading the meaning of "the".
\end{remark}

\begin{definition}
Let X be a set, $x: \mathbf{1} \longrightarrow X$ a function. We call $x$ an \textbf{element} of $X$ and write $x \in X$. For a function $f:X \longrightarrow Y$ we define the \textbf{evaluation} as a special case of composition 
\begin{equation*}
f(x) := f \circ x: \mathbf{1} \longrightarrow Y.
\end{equation*}
Notice that $f(x) \in Y$.
\end{definition}

\begin{axiom}[Empty Set]
There exists a set $\mathbf{0}$ with no elements.
\end{axiom}

\begin{remark}
The empty set will play the role of an initial object.
\end{remark}

\begin{axiom}[Equality of Functions]
Let $f,g: X \longrightarrow Y$ be two functions between sets. If $\forall x \in X: f(x) = g(x)$ then $f = g$.
\end{axiom}

\begin{definition}
A \textbf{generator} in a category $\mathcal{C}$ is an object $G$ such that for any two morphisms $f,g: X \longrightarrow Y$ in $\mathcal{C}$ if $f \neq g$ there exists a morphism $h: G \longrightarrow X$ such that $f \circ h \neq g \circ h$
\end{definition}

\begin{remark}
In terms of category theory, this means $\mathbf{1}$ is a generator in our category. It implies that $\mathbf{1}$ only has one element and our suggestive name is justified. We have enabled $\mathbf{1}$ to become the first tool we can use to examine the shape of a set.
\end{remark}

\begin{axiom}[Cartesian Product]
Every pair of sets $X, Y$ has a product $(X \times Y, p_1^{X,Y}, p_2^{X,Y})$.
\end{axiom}

\begin{axiom}[Functions as Set]
For every pair of sets $X, Y$, the exponential $Y^X$ exists.
\end{axiom}

\begin{remark}[$\dagger$]
The object $Y^X$ plays the role of an internal hom$[X,Y]$, in this case it wants to be the function set. For any fixed set $B$ we have an adjunction $(\-- \times B) \dashv (\--)^B$ and thus a natural bijection for any two other sets $A,C$:
\begin{equation*}
Hom(A \times B, C) \simeq Hom(A, C^B).
\end{equation*}

By Lawvere: For any two objects $A, B$ there exists an object $B^A$ and a mapping $A \times B^A \xrightarrow{e} B$ with the property that for any object $X$ and any mapping $A \times X\xrightarrow{f} B$ there is a unique mapping $X \xrightarrow{h} B^A$ such that $(\mathbf{1}_A \times h)e = f$. We call $e$ the $\textbf{evaluation map}$ and we have that for $a \in A, f:A\longrightarrow B$, we may evaluate the name $\fourIdx{\ulcorner\mkern-3mu}{}{\urcorner}{}{f} \in B^A$ as $(a,\fourIdx{\ulcorner\mkern-3mu}{}{\urcorner}{}{f})e = af$.

\end{remark}

\begin{definition}[$\dagger$]
Let $f: X \longrightarrow Y$ and $y \in Y$. We define the \textbf{Inverse Image} of $y$ under $f$ to be an object $A$ and a function $j:A \longrightarrow X$ such that:
\begin{enumerate}
\item $\forall a \in A: f(j(a)) = y$. Thus the following diagram must commute:

\begin{equation*}
\begin{tikzcd}
A \arrow[d, "j"]
& \mathbf{1} \arrow[d, "y"] \arrow[l, "a", swap] \\
 X \arrow[r, "f"]
& Y
\end{tikzcd}
\end{equation*}

\item For all objects $I$ and functions $q: I \longrightarrow X$ such that $\forall t \in I: f(q(t)) = y$ there exists a unique function $\bar{q}: I \longrightarrow A$ such that $q = j \circ \bar{q}$.

\begin{equation*}
\begin{tikzcd}
I 
\arrow[drr, bend left]
\arrow[ddr, bend right, "q"]
\arrow[dr, dotted, "{\bar{q}}" description] & & \\
& A \arrow[d, "j"] \arrow[r, ""]
& \mathbf{1} \arrow[d, "y"]  \\
& X \arrow[r, "f"]
& Y
\end{tikzcd}
\end{equation*}


\end{enumerate}
\end{definition}

\begin{axiom}[Inverse Image]
For every function $f: X \longrightarrow Y$, $y \in Y$, there exists an inverse image $f^{-1}(y)$ of $y$ under $f$.
\end{axiom}

\begin{definition}[$\dagger$]
An $\textbf{injection}$ is a function $j:A \longrightarrow X$ such that $j(a) = j(a') \implies a = a'$ for all $a,a' \in A$.
\end{definition}

\begin{definition}
A $\textbf{subset classifier}$ is a set $\mathbf{2}$ and an element $t \in \mathbf{2}$ such that the following holds:
\begin{enumerate}
\item For any sets $A,X$ and an injection $j: A \longrightarrow X$, there exists a unique function $\chi: X \longrightarrow 2$ such that $j: A \longrightarrow X$ is an inverse image of $t$ under $\chi$.
\end{enumerate}

\end{definition}

\begin{axiom}[Subsets]
There exists a subset classifier.
\end{axiom}

\begin{remark}
The previous axioms non-trivially imply that $\mathbf{2}$ is well-defined and in fact it is the two-element set we know. Notice that this means that we can now take pleasure in writing $\mathcal{P}(S) := \mathbf{2}^S$.
\end{remark}

\begin{definition}
A \textbf{natural number system} is a tuple $(N,0,s)$ with $N$ an object, $0 \in N$, and a $s: N \longrightarrow N$ such that for any object $X$, $a \in X$, and $r: X \longrightarrow X$
there is a unique $x: N \longrightarrow X$ such that the following diagram commutes:

\begin{equation*}
\begin{tikzcd}
\mathbf{1} \arrow[r, "0"] \arrow[d, "id_\mathbf{1}"]
& N \arrow[r, "s"] \arrow[d, "x", dotted]
& N \arrow[d, "x", dotted]   \\
\mathbf{1} \arrow[r, "a"]
& X \arrow[r, "r"]
& X
\end{tikzcd}.
\end{equation*}

\end{definition}

\begin{axiom}[Natural Numbers]
There exists a natural number system (Dedekind-Pierce Object).
\end{axiom}

\begin{remark}
The unique map $x$ maps $N$ onto a sequence beginning at $x_0 = a$ which is recursively defined through the function $r$ as $x_{n+1} = r(x_n)$. Thus we define our natural numbers through the shared behaviour of all such sequences (and their starting points), and it only matters how $N$ interacts with them.
\end{remark}

\begin{definition}[$\dagger$]
A $\textbf{surjection}$ is a function $s: X \longrightarrow Y$ such that for all $y \in Y$, there exists $x \in X$ with $s(x) = y$.
\end{definition}

\begin{definition}[$\dagger$]
A $\textbf{right inverse}$ of a function $s: X \longrightarrow Y$ is a function $i: Y \longrightarrow X$ such that $s \circ i = id_Y$.
\end{definition}

\begin{axiom}[Choice]
Every surjection has a right inverse.
\end{axiom}

\begin{remark}
This is equivalent to the Axiom of Choice, as for each $y \in Y$ we must choose an $x \in s^{-1}(y) \neq \emptyset$ for any surjection $s: X \longrightarrow Y$.
\end{remark}

\begin{remark}
The above axioms are weaker than ZFC, to have equivalence we require a final axiom. It allows for the existence of the universe created by $N$ and the power set function $\mathcal{P}(S) = \mathbf{2}^S$:
\begin{equation*}
\mathbf{N} \sqcup \mathcal{P}(\mathbf{N}) \sqcup \mathcal{P}(\mathcal{P}(\mathbf{N})) \sqcup ...
\end{equation*}
 and similar constructions. We state it informally:
\end{remark}

\begin{axiom}[Replacement]
Let $I$ be a set, and for each $i \in I$, let $X_i$ be specified (up to isomorphism) by a first order formula. Then there exists a set $X$ and a function $p: X \longrightarrow I$ such that $\forall i \in I: p^{-1}(i) \cong X_i$. 
\end{axiom}

\section{Subsets}

\begin{definition}
$a:X \longrightarrow A$ is a \textbf{subset} of $A$ if $a$ is a monomorphism.

$x$ is a \textbf{member} of $a$ if for some $A$, $x \in A$, $a$ is a subset of $A$, and there exists $\bar{x}$ such that $\bar{x}a = x$. In this case we also write $x \in a$.

We write $a \subseteq b$ if for some $A$, $a$ and $b$ are both subsets of $A$ and there exists $h$ such that $a = hb$ ie. $a$ factors over $b$.
\end{definition}

\begin{remark}
Instead of probing only with $\mathbf{1}$ we may now also examine more complex shapes through our monomorphisms.

Furthermore, the following proof follows from Lawvere's Axioms, where we have a slightly different axiom of choice which requires for any function $f$ (with a domain of at least one element), a semi-inverse $g$ such that $fgf = f$.
\end{remark}

\begin{theorem}
Let $a, b$ subsets of $A$. Then
\begin{equation}
a \subseteq b \iff \forall x \in A: x \in a \implies x \in b
\end{equation}
\end{theorem}

\begin{proof}
\begin{enumerate} 
\item $\Rightarrow :$ Let $a \subseteq b$ and $x \in A$ with $x \in a$. Then by definition we find a morphism $h$ and $\bar{x}$ such that their respective triangles commute:

\begin{equation}
\begin{tikzcd}[column sep=small]
& \mathbf{1} \arrow[dl, "\bar{x}"] \arrow[dr, dotted] \arrow[dd, "x", bend right=90,looseness=2,swap]& \\
\Box \arrow[rr, "h"] \arrow[dr, "a", hook] &     & \Box \arrow[dl, "b", hook] \\
& A
\end{tikzcd}
\end{equation} \newline

It follows that $x = \bar{x}a = \bar{x}(hb) = (\bar{x}h)b$ is our sought after factoring for $x \in b$.

\item $\Leftarrow :$ Let $a \in A$ and $a,b: \Box \longrightarrow A$ two monomorphisms. Then the Axiom of Choice implies that $\exists g: A \longrightarrow \Box : bgb = b$. By the left cancelative property of our monomorphism $b$, we retrieve $bg = id_\Box$. Define $h := ag$. We want to show that $a = hb = agb$. By Axiom 4 we may do so by proving that $\forall \bar{x} \in \Box: \bar{x}a = \bar{x}agb$. Fixing an arbitrary $\bar{x}$, we find that $x := \bar{x}a$ satisfies $x \in A$ with $x \in a$, thus $x \in b$ and it follows that $\exists y : x = yb$. Then
\begin{equation*}
\bar{x}hb = \bar{x}agb = xgb = (yb)gb = yb = x = \bar{x}a.
\end{equation*}
Thus $hb = a$ is our factoring for $a \subseteq b$. 
\end{enumerate}
\end{proof}





\clearpage


\begin{appendix}

\section{Small Things}

\begin{remark}
We say a morphism f factors over g if if there exists h such that $h \circ g = f$.
\end{remark}

\section{Sources}

\begin{enumerate}
\item AN ELEMENTARY THEORY
OF THE CATEGORY OF SETS (LONG VERSION)
WITH COMMENTARY - F. WILLIAM LAWVERE
\item Rethinking set theory - Tome Leinster
\item \href{https://golem.ph.utexas.edu/category/2014/01/an_elementary_theory_of_the_ca.html}{n-Category Cafe} - An Elementary Theory of the Category of Sets by Clive Newstead
\item ncatlab.org
\item Exploring Categorical Structuralism
COLIN MCLARTY
\end{enumerate}

\end{appendix}

\end{document}
